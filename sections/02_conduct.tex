
The conductivity $\sigma$ of the minerals in the nature covers a range of 25 decades! For example:
\begin{align*}
10^{-18} S/m &\rightarrow \textrm{Diamand}\\
10^{7} S/m &\rightarrow \textrm{Copper}\\
\end{align*}
Instead of the conductivity, the resistivity $\rho=\frac{1}{\sigma} \Omega m$ is often used.

\subsubsection*{Definition: Ohm's law}
Let us consider a rock sample of length $L$, resistivity $\rho$ and cross section $A$.
\begin{figure}[h!]
\begin{center}
\resizebox{0.4\textwidth}{!}
{
\begin{pspicture}(0,-2.5217187)(8.62,2.5217187)
\psellipse[linewidth=0.04,dimen=outer](1.2,-0.58171874)(0.6,1.0)
\psline[linewidth=0.04cm](1.2,0.41828126)(7.6,0.41828126)
\psline[linewidth=0.04cm](1.3,-1.5817188)(7.6,-1.5817188)
\psbezier[linewidth=0.04](7.6,-1.5817188)(8.1,-1.0817188)(8.1,-0.08171875)(7.6,0.41828126)
\psline[linewidth=0.04cm](8.0,-0.58171874)(8.6,-0.58171874)
\psline[linewidth=0.04cm](8.6,-0.58171874)(8.6,1.4182812)
\psline[linewidth=0.04cm](8.6,1.4182812)(4.9,1.4182812)
\psline[linewidth=0.04cm](4.3,1.4182812)(0.0,1.4182812)
\psline[linewidth=0.04cm](0.0,1.4182812)(0.0,-0.58171874)
\psline[linewidth=0.04cm](0.0,-0.58171874)(0.9,-0.58171874)
\usefont{T1}{ptm}{m}{n}
\rput(1.2460938,-0.57171875){\huge A}
\usefont{T1}{ptm}{m}{n}
\rput(4.1992188,-0.57171875){\huge I}
\usefont{T1}{ptm}{m}{n}
\rput(5.4471874,2.2282813){\huge U}
\usefont{T1}{ptm}{m}{n}
\rput(4.3992186,-2.2717187){\huge L}
\pscircle[linewidth=0.04,dimen=outer](4.6,1.4182812){0.3}
\psline[linewidth=0.04cm,arrowsize=0.05291667cm 2.0,arrowlength=1.4,arrowinset=0.4]{->}(4.7,-2.2817187)(7.6,-2.2817187)
\psline[linewidth=0.04cm,arrowsize=0.05291667cm 2.0,arrowlength=1.4,arrowinset=0.4]{->}(4.1,-2.2817187)(1.2,-2.2817187)
\end{pspicture} 
}

\caption{Schematic derivation of Ohm's law}
\label{fig:ohmslaw}
\end{center}
\end{figure}

A current $I [A]$ flows by applying a voltag $U [V]$ to the rock sample:

\begin{align*}
I&=\frac{AU}{\rho L}\\
\Leftrightarrow \rho\underbrace{\frac{I}{A}}_{\textrm{current density }j}&=\underbrace{\frac{U}{L}}_{\textrm{electric Field }E}
\end{align*}
\begin{equation}
\vec{j}\rho=\vec{E}\label{eq:1-2}
\end{equation}
We measure $I$ and $U$, $A$ and $L$ are known, so we can calculate $\rho$.

\subsection{Mechanisms of electrical conductivity}
\begin{description}
\item[Metallic conductivity:] Current flows by free electrons $\rho\equiv T$
\item[Electrolytic conductivity:] Charge carriers are cations and anions: $\rho$ decreases with temperature $T$.
\item[Semi-conductors:] Charge carriers must be activated by heat, light or EM-radiation. Strongly dependent on temperature $T$. Important for mantle (deep earth structures)
\item[Boundary layer conductivity:] Occurs due to the interaction of the pore liquid with the rock matrix. This is the source of SP-anomalies!
\end{description}

