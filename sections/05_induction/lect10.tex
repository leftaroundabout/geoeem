\setcounter{section}{5}\setcounter{subsection}{1}\addtocounter{subsection}{-1}\subsection{Field station }

\begin{itemize}
\tightlist
\item
  3 component magnetometer
\item
  For long periods \(10^{-6}-1 \physu{Hz}\) use of flux gate
  magnetometers
\item
  For short periods \(10^{-3}-10^3 \physu{Hz}\) use induction coil
  magnetometers
\item
  Non-polarizable electrodes
\item
  e.g.~Ag-AgCl have nearly no disturbing potential
\end{itemize}

\setcounter{section}{5}\setcounter{subsection}{2}\addtocounter{subsection}{-1}\subsection{Analysis of MT time series }

Overview of different steps:

Choice of sensors \(\to\) Observation \(\to\) Pretreatment of time
series in time domain \(\to\) Transform the data to frequency domain
\(\to\) Response curve \(\to\) Power spectra, stacking \(\to\)
Calculation of the transfer functions
(\(Z_{xx}, Z_{xy}, Z_{yx}, Z_{yy}, T_x, T_y\))

\begin{enumerate}
\def\labelenumi{\arabic{enumi}.}
\item
  Choice of the frequency range and magnetometer: indution coil or flux
  gate? Determination of the sampling interval \(\Delta t\)
\item
  Observation. Measure \(E_x(t)\), \(E_y(t)\), \(H_x(t)\), \(H_y(t)\),
  \(H_z(t)\)

  \begin{itemize}
  \tightlist
  \item
    Period: 1 d \ldots{} 1 yr
  \item
    Digitalisation \(H(t_j) \to x_j\), \(E(t_j) \to z_j\)
  \end{itemize}
\item
  Choice of analysing intervals:

  \begin{itemize}
  \tightlist
  \item
    Intervals with strong magnetic activity but without visible noise
  \item
    Choose time intervals suitable for FFT
  \end{itemize}
\item
  Data processing in time domain

  \begin{itemize}
  \tightlist
  \item
    Determine the trend of each chosen time interval and substract it
  \item
    Numerical filter

    \begin{itemize}
    \tightlist
    \item
      Aim: distinguish or emphasize suitable frequency bands
    \item
      Preparation for the frequency analysis
      \[ y_n = \sum_{n'=-N}^{N} w_{n'}x_{n-n'} \] high pass, low pass
      filters. Consider low pass filter, e.g.
    \item
      rectangle/sinc LP filter \[
         \tilde{w}(\omega) = \begin{cases}
                1   & |\omega|<\omega_0
           \\ (1/2  & |\omega|=\omega_0 )
            \\  0   & \text{otherwise}.
            \end{cases}
       \] This has difficulties in practice due to strong oscillations
      in the vicinity of cutoff frequency.
    \item
      Trapezoidal LP filter \[
         \tilde{w}(\omega) = \begin{cases}
                1   & 0<|\omega|<\omega_1
              \frac{\omega_2-\omega}{\omega_2-\omega_1}
           \\       & \omega_1 < |\omega| < \omega_2
            \\  0   & \text{otherwise}.
            \end{cases}
       \] The following quantities are introduced as filter parameters:
      \[
         \omega_0 = \tfrac12(\omega_1 + \omega_2)
       \] \[
         \Omega = \tfrac12(\omega_2 - \omega_1).
       \] The \emph{steepness} of the filter is defined by \[
         S = \frac{\omega_0}{\Omega}.
       \] The filter function \(w(t)\) can be obtained by applying the
      FT on \(\tilde{w}(\omega)\) \[
         \omega(t) = \tfrac1\pi \int\limits_0^{\infty}\ttd{\omega} \tilde{w}\cos(\omega t)
                   = \frac{\omega_0}{\phi}\frac{\sin(\omega_0 t)}{\omega_0t}
                       \cdot\frac{\sin(\Omega t)}{\Omega t}.
       \] The filter should have finite length, so define
      \(\tilde{w}(\omega)\): \[
         \tilde{w}_m(\omega) = \tilde{w}_0 + 2\sum_{n=1}^K w_n\cos(2\pi nm/N)
       \] with \[
         \tilde{w}_0 = \frac{\omega_0}{\omega\rmsc{ny}}
             \left(1 + 2\frac{\sin\alpha_n}{\alpha_n}\cdot\frac{\sin\beta_n}{\beta_n}\right)
       \]
    \item
      Trapezoidal HP filter \[
         \tilde{w}\rmsc{H} = 1 - \tilde{w}\rmsc{L}.
       \] Instead of high pass, a polynomial equation can be applied.
      (???) \[
         y_n' = \frac{d}{T}(t - t_0 - \tfrac{\tau}{2})
       \] with \(d = y(t_N) - y(t_0)\). The analysis can be realized in
      small and overlapping frequency ranges. Example of typical filter
      parameters
    \item
      Magnetic storm, \(T = 8\physu{d}\). \(\Delta t\sim 1 h\). Low pass
      \(0.75\physu{cpd}\), polynomial high pass.
    \item
      Day variation, \(T = 1\physu{d}\). \(\Delta t\sim 1 h\). Low pass
      \(4\physu{cpd}\), polynomial high pass.
    \item
      Variations, \(T = 6\physu{h}\). \(\Delta t\sim 1 \physu{min}\).
      Low pass \(2\physu{cpd}\), polynomial high pass.
    \item
      Variations, \(T = 6\physu{h}\). \(\Delta t\sim 1 \physu{min}\).
      Low pass \(6\physu{cpd}\), high pass \(0.5\physu{cpd}\).
    \item
      Variations, \(T = 6\physu{h}\). \(\Delta t\sim 1 \physu{min}\).
      Low pass \(15\physu{cpd}\), high pass \(1.5\physu{cpd}\).
    \end{itemize}
  \end{itemize}
\item
  Data processing in frequency domain

  \begin{itemize}
  \tightlist
  \item
    Harmonic analysis of the filtered time series \[
       x_j , j\in\{0\ldots N\}
        \Rightarrow \tilde{X}_m, m\in\{0\ldots M/2\}
     \] Outcome: raw spectra. \[
       x_j = \sum_{m=0}^{M} (a_m\cos(m\phi_j) + b_m\sim(m\phi_j) + S x_j
     \] with \(\phi_j = \frac{2\pi j}{N}\). \[
       a_m = \frac{2}N \sum_{j=0}^{N-1} x_j\cos(m\phi_j)
     \] \[
       b_m = \frac{2}N \sum_{j=0}^{N-1} x_j\sin(m\phi_j)
     \] \[
       a_0 = \tfrac1N \sum_{j=0}^{N-1} x_j
     \] \[
       a_M = \tfrac1N \sum_{j=0}^{N-1} x_j(-1)^j
     \] The corresponding frequency \[
       \omega_m = \frac{2\pi}{T} = \frac{\pi}\Delta
     \] is called the Nyqvist frequency. The analysis for
    \(\omega > \omega_M\) is not unique, we get \emph{aliasing}. In the
    harmonic analysis only those frequencies are considered which are
    less than the Nyqvist frequency. It is assumed that the time series
    vanishes outside the transformation interval, but this must be
    enforced by multiplication with a \emph{window function}. This is
    equivalent to a convolution in frequency domain. Example of a window
    functions:

    \begin{itemize}
    \tightlist
    \item
      Hann, \(w(t) = \tfrac12 (1 + \cos\tfrac{2\pi}{T}(t - t_0 - T/2))\)
      Different possibilities to calculate harmonic coefficients.
    \end{itemize}
  \item
    Calibration: at this stage the response curve of the device is
    considered. For example, with
  \item
    \(g(t)\) observed time series
  \item
    \(f(t)\) true time series
  \item
    \(r(t)\) response function of the device \[
    g(t) = r(t) \ast f(t)
       \] Example: measurement of the earth magnetic field pulsation
    using an induction coil magnetometer.
  \item
    Time dom. \(g(t) = r(t)\ast \partial_t f(t)\)
  \item
    Freq dom. \(G(\omega) = i\omega F(\omega)\)
  \end{itemize}
\end{enumerate}
