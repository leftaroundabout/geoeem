
\begin{figure}[h!]
\begin{center}
\resizebox{0.5\textwidth}{!}
{
\begin{pspicture}(0,-2.0159376)(15.188437,2.0159376)
\psline[linewidth=0.04cm](0.1684375,-1.0303125)(15.168438,-1.0303125)
\psline[linewidth=0.04cm,arrowsize=0.05291667cm 2.0,arrowlength=1.4,arrowinset=0.4]{->}(3.1684375,0.9696875)(3.1684375,-1.0303125)
\psline[linewidth=0.04cm,arrowsize=0.05291667cm 2.0,arrowlength=1.4,arrowinset=0.4]{->}(6.1684375,-0.0303125)(6.1684375,-1.0303125)
\psline[linewidth=0.04cm,arrowsize=0.05291667cm 2.0,arrowlength=1.4,arrowinset=0.4]{->}(9.168438,-0.0303125)(9.168438,-1.0303125)
\psline[linewidth=0.04cm,arrowsize=0.05291667cm 2.0,arrowlength=1.4,arrowinset=0.4]{->}(12.168438,0.9696875)(12.168438,-1.0303125)
\psline[linewidth=0.04cm](6.1684375,-0.0303125)(7.4684377,-0.0303125)
\psline[linewidth=0.04cm](7.9684377,-0.0303125)(9.168438,-0.0303125)
\psline[linewidth=0.04cm](3.1684375,0.9696875)(8.168438,0.9696875)
\psline[linewidth=0.04cm](9.168438,0.9696875)(12.168438,0.9696875)
\pscircle[linewidth=0.04,dimen=outer](7.7184377,0.0196875){0.25}
\pscircle[linewidth=0.04,dimen=outer](8.668438,0.9696875){0.5}
\psline[linewidth=0.04cm,arrowsize=0.05291667cm 2.0,arrowlength=1.4,arrowinset=0.4]{->}(7.4684377,-0.2303125)(8.068438,0.3696875)
\psline[linewidth=0.04cm,arrowsize=0.05291667cm 2.0,arrowlength=1.4,arrowinset=0.4]{->}(8.168438,0.4696875)(9.268437,1.6696875)
\usefont{T1}{ptm}{m}{n}
\rput(7.9475,-0.4553125){\LARGE $\Delta U$}
\usefont{T1}{ptm}{m}{n}
\rput(7.974375,1.7446876){\LARGE Source $I$}
\usefont{T1}{ptm}{m}{n}
\rput(3.179375,-1.4553125){\LARGE A}
\usefont{T1}{ptm}{m}{n}
\rput(6.223906,-1.4553125){\LARGE M}
\usefont{T1}{ptm}{m}{n}
\rput(9.084687,-1.4553125){\LARGE N}
\usefont{T1}{ptm}{m}{n}
\rput(12.147187,-1.4553125){\LARGE B}
\usefont{T1}{ptm}{m}{n}
\rput(1.4175,-1.7553124){\LARGE $\rho$}
\usefont{T1}{ptm}{m}{n}
\rput(1.5075,-0.4553125){\LARGE $\rho=\infty$}
\end{pspicture} 
}

\caption{Four point measurement}
\label{fig:dc01}
\end{center}
\end{figure}
Resistivity $\rho$ of the subsurface derived from $I$ (which is known), $\Delta U$ (which is measured) and the geometrical factor $K$ (which is also known).

\subsubsection*{Frequently used electrode arrays}

\begin{figure}[h!]
\begin{center}
\includegraphics[width=0.5\textwidth]{Wenner.png}
a)
\includegraphics[width=0.5\textwidth]{Schlumberger.png}
b)
\includegraphics[width=0.5\textwidth]{Dipoldipol.png}
c)
\caption{a) Wenner, Half-Wenner; b) Schlumberger, Half-Schlumberger; c) Dipole-dipole, source???}
\label{fig:dc02}
\end{center}
\end{figure}

Industrial standard of measuring is via an \textit{Multielectrode array}.


\subsection{Basic equations of DC-resistivity}
The first assumption of DC-resistivity methods and the major difference to EM-methods is the assumption of stationary currents:
\begin{align*}
\frac{\partial}{\partial t}=0
\end{align*}
The fields do not depend on time.

Looking at the \textit{Maxwell's equations}:
\begin{align}
\nabla\times\vec{E}=\frac{\partial \vec{B}}{\partial t}=0
\end{align}
This means irrotational electric field and from that follows, that the electric field vector can be derived by a scalar potential:

\begin{equation}
\vec{E}=-\nabla V \label{eq:2-2}
\end{equation}
Insert equation \eqref{eq:2-2} into eq. \eqref{eq:1-2}:

\begin{equation}
\vec{j}=-\sigma \nabla V\label{eq:2-3}
\end{equation}
\textit{Continuity equation:}
\begin{equation}
\nabla\cdot\vec{j}+\frac{\partial q}{\partial t}=0
\end{equation}
Now new charges are generated in the course of time
\begin{equation}
\nabla\cdot\vec{j}=0 \label{eq:2-5}
\end{equation}
which is valid outside of the source.

If we insert eq. \eqref{eq:2-3} into \eqref{eq:2-5}:
\begin{align*}
-\nabla\cdot(\sigma\nabla V)&=0\\
\nabla\sigma\nabla V + \sigma\nabla^2V&=0
\end{align*}
$\nabla\sigma=0$ for areas with constant conductivity, so:
\begin{equation}
\nabla^2 V=0\label{eq:lapleq}
\end{equation}
which is called the \textit{Laplace-equation}, the basic equation of DC-resistivity.

Derivation of solutions of this elliptic partial differential equation using  different boundary conditions:

Assume a current source with strength $I$ at point $\vec{r}_0$, then the spatial current distribution can be given as:$\nabla\cdot\vec{j}=I\delta(\vec{r}-\vec{r}_0$
and so:
\begin{equation}
\nabla\cdot(\sigma\nabla V)=-I\delta(\vec{r}-\vec{r}_0
\end{equation}
This equation can be solved numerically for arbitrary distribution of conductivity ratio.


\subsubsection{Potential of a current electrode}

\begin{figure}[h!]
\begin{center}
\resizebox{0.5\textwidth}{!}
{
\begin{pspicture}(0,-3.02)(10.189688,1.02)
\psline[linewidth=0.04cm](0.0,0.0)(8.0,0.0)
\rput{-180.0}(8.0,0.0){\psarc[linewidth=0.04,linestyle=dashed,dash=0.16cm 0.16cm](4.0,0.0){3.0}{0.0}{180.0}}
\psline[linewidth=0.04cm,arrowsize=0.05291667cm 2.0,arrowlength=1.4,arrowinset=0.4]{->}(4.0,1.0)(4.0,0.0)
\psline[linewidth=0.04cm,arrowsize=0.05291667cm 2.0,arrowlength=1.4,arrowinset=0.4]{->}(4.0,0.0)(1.2,-1.0)
\psline[linewidth=0.04cm,arrowsize=0.05291667cm 2.0,arrowlength=1.4,arrowinset=0.4]{->}(4.0,0.0)(4.0,-3.0)
\psline[linewidth=0.04cm,arrowsize=0.05291667cm 2.0,arrowlength=1.4,arrowinset=0.4]{->}(4.0,0.0)(6.9,-1.0)
\psline[linewidth=0.04cm,arrowsize=0.05291667cm 2.0,arrowlength=1.4,arrowinset=0.4]{->}(4.0,0.0)(1.7,-1.9)
\psline[linewidth=0.04cm,arrowsize=0.05291667cm 2.0,arrowlength=1.4,arrowinset=0.4]{->}(4.0,0.0)(6.3,-1.9)
\psline[linewidth=0.04cm,arrowsize=0.05291667cm 2.0,arrowlength=1.4,arrowinset=0.4]{->}(4.0,0.0)(2.7,-2.7)
\psline[linewidth=0.04cm,arrowsize=0.05291667cm 2.0,arrowlength=1.4,arrowinset=0.4]{->}(4.0,0.0)(5.3,-2.7)
\usefont{T1}{ptm}{m}{n}
\rput(8.788125,-0.595){equipotential surface}
\usefont{T1}{ptm}{m}{n}
\rput(2.1721876,-0.195){current flow}
\psline[linewidth=0.04cm,arrowsize=0.05291667cm 2.0,arrowlength=1.4,arrowinset=0.4]{->}(7.2,-0.6)(7.0,-0.5)
\psline[linewidth=0.04cm,arrowsize=0.05291667cm 2.0,arrowlength=1.4,arrowinset=0.4]{->}(1.7,-0.4)(1.8,-0.7)
\rput{-180.0}(8.0,0.0){\psarc[linewidth=0.04,linestyle=dashed,dash=0.16cm 0.16cm](4.0,0.0){2.0}{0.0}{180.0}}
\end{pspicture} 
}
\caption{Single current source}
\label{fig:singlesource}
\end{center}
\end{figure}

Using \textit{Ohm's law}: $\vec{E}=\rho\vec{j}=\rho\frac{I}{2\pi r^2}$, where $2\pi r^2$ is the surface of the half sphere. Using $E=-\frac{dV}{dr}$ follows the potential of a homogeneous half space:
\begin{equation}
V=\frac{\rho I}{2\pi r}
\end{equation}

\begin{figure}[h!]
\begin{center}
\resizebox{0.5\textwidth}{!}
{
\begin{pspicture}(0,-2.9690626)(13.175938,2.9890625)
\psline[linewidth=0.04cm](0.0,1.0509375)(8.0,1.0509375)
\psline[linewidth=0.04cm](2.0,-1.9490625)(4.0,2.0509374)
\psline[linewidth=0.04cm](4.0,2.0509374)(5.2,2.0509374)
\psline[linewidth=0.04cm](6.0,2.0509374)(6.8,2.0509374)
\psline[linewidth=0.04cm,linestyle=dashed,dash=0.16cm 0.16cm,arrowsize=0.05291667cm 2.0,arrowlength=1.4,arrowinset=0.4]{->}(6.8,2.0509374)(9.3,2.0509374)
\usefont{T1}{ptm}{m}{n}
\rput(10.955313,2.3959374){\Large $c_2\rightarrow\infty$}
\psdots[dotsize=0.16](2.0,-1.9490625)
\psline[linewidth=0.04cm,arrowsize=0.05291667cm 2.0,arrowlength=1.4,arrowinset=0.4]{->}(2.0,-1.9490625)(3.0,-1.9490625)
\psline[linewidth=0.04cm,arrowsize=0.05291667cm 2.0,arrowlength=1.4,arrowinset=0.4]{->}(2.0,-1.9490625)(2.0,-0.9490625)
\psline[linewidth=0.04cm,arrowsize=0.05291667cm 2.0,arrowlength=1.4,arrowinset=0.4]{->}(2.0,-1.9490625)(1.0,-1.9490625)
\psline[linewidth=0.04cm,arrowsize=0.05291667cm 2.0,arrowlength=1.4,arrowinset=0.4]{->}(2.0,-1.9490625)(2.0,-2.9490626)
\pscircle[linewidth=0.04,linestyle=dashed,dash=0.16cm 0.16cm,dimen=outer](2.0,-1.9490625){1.0}
\psline[linewidth=0.04cm,arrowsize=0.05291667cm 2.0,arrowlength=1.4,arrowinset=0.4]{->}(2.0,-1.9490625)(2.7,-2.6490624)
\psline[linewidth=0.04cm,arrowsize=0.05291667cm 2.0,arrowlength=1.4,arrowinset=0.4]{->}(2.0,-1.9490625)(1.3,-2.6490624)
\psline[linewidth=0.04cm,arrowsize=0.05291667cm 2.0,arrowlength=1.4,arrowinset=0.4]{->}(2.0,-1.9490625)(1.3,-1.2490625)
\psline[linewidth=0.04cm,arrowsize=0.05291667cm 2.0,arrowlength=1.4,arrowinset=0.4]{->}(2.0,-1.9490625)(2.7,-1.3490624)
\usefont{T1}{ptm}{m}{n}
\rput(3.5653124,-1.7040625){\Large $c_1$}
\pscircle[linewidth=0.04,dimen=outer](5.6,2.0509374){0.4}
\psline[linewidth=0.04cm,arrowsize=0.05291667cm 2.0,arrowlength=1.4,arrowinset=0.4]{->}(5.3,1.6509376)(6.0,2.5509374)
\usefont{T1}{ptm}{m}{n}
\rput(5.6876564,2.7809374){\large Source}
\end{pspicture} 
}

\caption{Mise a la masse method}
\label{fig:misemasse}
\end{center}
\end{figure}

In the case of the \textit{Mise a la masse method} the potential of the homogeneous full space is:
\begin{equation}
V=\frac{\rho I}{4\pi r}
\end{equation}

The same result can be derived by using the Laplace-equation \eqref{eq:lapleq} and the use of spherical coordinates:
\begin{align*}
\nabla^2 V=\frac{d^2V}{dr^2}+\frac{2}{r}\frac{dV}{dr}
\end{align*}
From the symmetry of the system the potential is a function of the distance to the source $r$ only. Multiplying by $r^2$ and integrating, we get:
\begin{align*}
\frac{dV}{dr}=\frac{c_1}{r^2}
\end{align*}
Integrating over $r$ again leads to the solution:
\begin{align*}
V=-\frac{c_1}{r}+c_2 && c_1,c_2 = const.
\end{align*}
To determine the constants we have to use boundary conditions: From $\lim_{r \to \infty} V(r) = 0$ follows that $c_2=0$. Using the current density: $j=\frac{I}{A}\Leftrightarrow I=jA$:
\begin{equation*}
I=4\pi r^2j=-4\pi r^2\sigma\frac{dV}{dr}=-4\pi\sigma c_1
\end{equation*}
From this equation we can derive $c_1$:
\begin{equation}
V=\frac{I\rho}{4\pi r}
\end{equation}

\subsection*{Boundary equations}
Boundary with different conductivities.

\begin{figure}
\begin{center}
\includegraphics[scale=0.5]{figs/grenz.pdf}
\caption{Boundary with dip angles GEOING s5}
\label{fig:bound01}
\end{center}
\end{figure}

Two boundary conditions which must hold at any contact between two regions of different conductivity.
\begin{itemize}
\item Potential is continuous across the boundary
\item $j_n$ is also continuous.
\end{itemize}
\begin{align*}
V^1=V^2, ~~~\left(\frac{\partial V}{\partial x}\right)^1=\left(\frac{\partial V}{\partial x}\right)^2,~~~ j_n^1=j_n^2
\end{align*}
\begin{align*}
E_t^1=E_t^2, ~~~\sigma_1 E_n^1=\sigma_2 E_n^2
\end{align*}
\begin{align*}
\sigma_1 \frac{E_n^1}{E_t^1}&=\sigma_2 \frac{E_n^2}{E_t^2}\\
\sigma_1\cot\alpha&=\sigma_2\cot\beta\\
\frac{\tan\alpha}{\tan\beta}&=\frac{\sigma_1}{\sigma_2}
\end{align*}
Current line is bent towards to the normal if the resistivity of the second medium $\rho_2$ is larger than the one of the first medium $\rho_1$.

\begin{figure}[h!]
\begin{minipage}{0.45\textwidth}
	\centering
	\includegraphics[width=\textwidth]{grenz_02.eps}
\end{minipage}
\hspace{0.05\textwidth}
\begin{minipage}{0.45\textwidth}
\centering
	\includegraphics[width=\textwidth]{grenz_03.eps}
\end{minipage}
\begin{minipage}[t]{0.45\textwidth}
\centering
	\captionof{figure}{asdf1}
	\label{bound_01}
\end{minipage}
\hspace{0.05\textwidth}
\begin{minipage}[t]{0.45\textwidth}
	\centering
	\captionof{figure}{asdf2}
	\label{bound_02}
\end{minipage}
\end{figure}

\subsubsection{Potential distribution at the surface of a horizontally stratified earth (Solution of the Laplace equation \eqref{eq:lapleq})}
Starting with a \textit{model}:

\begin{figure}[h!]
\begin{center}
\resizebox{0.3\textwidth}{!}
{
\begin{pspicture}(0,-3.0375)(6.541875,3.0175)
\psline[linewidth=0.04cm](0.0,1.6975)(4.0,1.6975)
\psline[linewidth=0.04cm](0.0,0.6975)(4.0,0.6975)
\psline[linewidth=0.04cm](0.0,-0.3025)(4.0,-0.3025)
\psline[linewidth=0.04cm](0.0,-2.3025)(4.0,-2.3025)
\usefont{T1}{ptm}{m}{n}
\rput(0.95734376,1.2275){\large $\rho_1$}
\usefont{T1}{ptm}{m}{n}
\rput(0.95734376,0.2275){\large $\rho_2$}
\usefont{T1}{ptm}{m}{n}
\rput(0.94734377,-2.7725){\large $\rho_n$}
\usefont{T1}{ptm}{m}{n}
\rput(3.4173439,-2.7725){\large $h_n\rightarrow\infty$}
\usefont{T1}{ptm}{m}{n}
\rput(3.0173438,0.2275){\large $h_2$}
\usefont{T1}{ptm}{m}{n}
\rput(3.0173438,1.2275){\large $h_1$}
\psline[linewidth=0.04cm,arrowsize=0.05291667cm 2.0,arrowlength=1.4,arrowinset=0.4]{->}(2.0,2.6975)(2.0,1.6975)
\psline[linewidth=0.04cm](2.0,2.6975)(3.2,2.6975)
\psline[linewidth=0.04cm](3.2,2.8975)(3.2,2.4975)
\psline[linewidth=0.04cm](3.4,2.9975)(3.4,2.3975)
\psline[linewidth=0.04cm](3.4,2.6975)(4.0,2.6975)
\psline[linewidth=0.04cm,linestyle=dashed,dash=0.16cm 0.16cm,arrowsize=0.05291667cm 2.0,arrowlength=1.4,arrowinset=0.4]{->}(4.0,2.6975)(5.5,2.6975)
\usefont{T1}{ptm}{m}{n}
\rput(5.9414062,2.7025){$\infty$}
\psdots[dotsize=0.1](1.0,-0.8025)
\psdots[dotsize=0.1](1.0,-1.1025)
\psdots[dotsize=0.1](1.0,-1.4025)
\psdots[dotsize=0.1](3.0,-0.8025)
\psdots[dotsize=0.1](3.0,-1.1025)
\psdots[dotsize=0.1](3.0,-1.4025)
\end{pspicture} 
}

\caption{Model of $n$ layer structure}
\label{fig:model01}
\end{center}
\end{figure}

The subsurface consists of finite number of layers with the last layer having infinite layer thickness ( $h_n\rightarrow\infty$ ). We assume that $\rho_i$ is isotropic (no dependence of the direction of measurement). The field is generated by a point source with the current $I$ is a direct current.

Starting from the Laplace equation with potential $V$:
\begin{equation}
\frac{\partial^2 V}{\partial x^2}+\frac{\partial^2 V}{\partial y^2}+\frac{\partial^2 V}{\partial z^2}=0
\end{equation}
In cylindrical coordinates ($r,\theta,z$):
\begin{equation}
\frac{\partial^2 V}{\partial r^2}+\frac{1}{r}\frac{\partial V}{\partial r}+\frac{\partial^ V}{\partial z^2}+\frac{1}{r^2}\frac{\partial^2 V}{\partial\theta^2}=0
\end{equation}

The solution is symmetrical to the vertical axis, so $\frac{\partial V}{\partial \theta}=\frac{\partial^2 V}{\partial\theta^2}=0$, so $V(r,\theta,z)=V(r,z)$. So the Laplace equation to be solved reduces to:
\begin{equation}
\frac{\partial^2 V}{\partial r^2}+\frac{1}{r}\frac{\partial V}{\partial r}+\frac{\partial^ V}{\partial z^2}=0
\end{equation}

