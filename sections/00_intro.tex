Common to each method is the fact that the current flow is used in the subsurface. The aim is the determination of the conductivity distribution of the subsurface from the Earth's surface down to several 100 km depth.

Application areas:
\begin{itemize}
\item Near surface exploration (0 - 300 m depth): 
\begin{itemize}
\item Application for the environment: Waste site exploration, search for suitable landfill sites, ...
\item Groundwater exploration
\item Archaeology
\item Exploration for deposits, engineering applications (e.g. cativity detection,...)
\end{itemize}
\item Exploration of deep structures ( $>$ 300 m)
\begin{itemize}
\item Geothermal fields, oil and gas exploration
\item tectonic questions, shear zones
\item deep crust and upper mantle
\end{itemize}


\end{itemize}
\subsection{Classification of methods}
Classifications possible as:
\begin{itemize}
\item According to the source (artificial or natural)
\item Inclusion of magnetic field or not?
\item Direct current or alternating current?
\end{itemize}
\begin{description}
\item[DC-resistivity methods:] Direct current resistivity (DC), Induced polarization (IP), Self potential (SP)
\item[Electromagnetic methods:] ~\\
\begin{itemize}
\item \textit{Frequency domain}: Magnetotellurics (MT), Audiomagnetotellurics (AMT), Controlled source AMT (CSAMT), Radiomagnetotellurics (RMT)
\item \textit{Time domain:} Transient electromagnetics (TEM), Long offset transient electromagnetics (LOTEM)
\end{itemize}
\item[Electromagnetic methods using high frequencies ($f > 10$ MHz):] Ground penetrating radar (GPR)
\end{description}
