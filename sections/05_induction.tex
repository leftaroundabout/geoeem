\subsection{Layered models}
General solution approach of eq. 3.13.
\begin{align*}
\frac{d^2\hat{E}_x}{dz^2}=(i\omega\mu_0\sigma+k^2)\hat{E}_x
\end{align*}
in the $\omega,k$ domain for TE-fields $(E_z=0)$ for the m. layer.

\begin{align*}
\hat{E}_x(z)=A_me^{-K_m z}+B_me^{K_m z}&& z_m<z<z_{m+1}
\end{align*}

with 
\begin{align}
K_m=\sqrt{i\omega\mu_0\sigma+k^2}
\end{align}

\begin{align*}
\hat{B}_z=k(Ee^{-kz}-Ie^{kz})
\end{align*}


Using eq. 3.15 ($\hat{B}_y=-\frac{1}{i\omega}\frac{d\hat{E}_x}{dz}$):

\begin{align}
\hat{B}_y=\frac{-1}{i\omega}\frac{d\hat{E}_x}{dz}=\frac{K_m}{i\omega}(A_me^{-K_m z}+B_me^{K_m z})
\end{align}

For the homogeneous half space of the model:

\begin{align*}
\hat{E}_x(z)=A_me^{-K_m z} && \hat{B}_y(z)=\frac{K_m}{i\omega}A_m e^{-K_m z}
\end{align*}

Analogous for $\hat{E}_y$ and $\hat{B}_x$ (see chapter 3.4).

Continuity of $\hat{E}_x$ and $\hat{B}_y$ at the layer boundaries are valid if their impedance ratio:

\begin{align}
\frac{\hat{E}_x}{\hat{B}_y}=\frac{i\omega}{K_mG(z)}
\end{align}

with 
\begin{align}
G(z)=\frac{A_me^{-K_m z}-B_me^{K_m z}}{A_me^{-K_m z}+B_me^{K_m z}}
\end{align}

is continuous. $G(z)$ changes as the ratio of the vertical wave numbers.

\begin{align*}
K_m \underbrace{G_M^+}_{G(z_{m+1}^{-0})}=K_{m+1}\underbrace{G_{m+1}^-}_{G(z_{m+1}^{+0})}
\end{align*}
for $z=z_{m+1}$.
\begin{align*}
G_m^+=\frac{K_{m+1}}{K_m}G_{m+1}^-
\end{align*}
For the half space of the model:
\begin{align*}
G_{m}^-=1
\end{align*} 

For the connection to the air region: $K_0G_0^+=K_1G_1^-$.

For the connection $G_m^-$ to $G_m^+$:
\begin{align}
K_mz_{m+1}=K_mz_m+K_md_m
\label{eq:eq6-5}
\end{align}
with $d_m=z_{m+1}-z_m$ the thickness of the $m$'th layer.

\begin{align*}
\Rightarrow e^{\pm K_mz_{m+1}}=e^{\pm\alpha}\left(\cosh(\beta)\pm\sinh(\beta)\right)
\end{align*}
Now insert in the eq. 3.4 $(G(z) = ...)$ for $z=z_{m+1}^{-0}$

\begin{align}
G_m^+=\frac{(A_me^{-\alpha}-\beta e^\alpha)\cosh(\beta)-(A_me^{-\alpha}+\beta e^\alpha)\sinh(\beta)}{(A_me^{-\alpha}+\beta e^\alpha)\cosh(\beta)-(A_me^{-\alpha}-\beta e^\alpha)\sinh(\beta)}
\end{align}

Divide by $(A_me^{-\alpha}+\beta e^\alpha)\cosh(\beta)$

\begin{align*}
G_m^+=\frac{G_m^--\tanh(\beta)}{1-G_m^-\tanh(\beta)}
\end{align*}
or 
\begin{align*}
G_m^+=\frac{G_m^++\tanh(\beta)}{1+G_m^+\tanh(\beta)}
\end{align*}
Inserting of eq. \eqref{eq:eq6-5} yields the \textit{Recursion formula of WAIT}:

\begin{align}
G_m^-=\frac{K_{m+1}G_{m+1}^-+K_m\tanh(K_md_m)}{K_m+K_{m+1}G_{m+1}^-\tanh(K_md_m)}
\end{align}


Steps to do:
Successive selection for a given ground model. For $m=M-1,M-2,...,2,1$, $G_M^-=1$ and determination of $G_1^-$ for the ground surface at $z=\pm 0$. If $G_1^-$ is determined, calculate impedance $z$ or complex penetration depth $C$ according to
\begin{align*}
\frac{\hat{E}_x}{\hat{E}_y}=\frac{i\omega}{K_mG(z)}
\end{align*}
Then 
\begin{align}
\hat{E}_x(0,k,\omega)=i\omega C(k,\omega)\hat{B}_y(0,k,\omega)
\end{align}
with $C(k,\omega)=(G_1^-K_1)^{-1}$

Extended definition compared to eq. 2.18:

\begin{align*}
C(k,\omega)=K^{-1}=\frac{p}{2}(1-i)
\end{align*}
\begin{align}
C(k,\omega)=\frac{\hat{E}_x(0,k,\omega)}{-\frac{d\hat{E}_x}{dz}\bigg|_{z=0}}=\frac{\hat{E}_x}{i\omega\hat{B}_y}
\end{align}

\newcommand{\Ehat}{\hat{E}}
\newcommand{\Bhat}{\hat{B}}

\subsubsection*{Analogous solution approach for $B_y$ for the induction by TM fields:}
\begin{align*}
\hat{B}_y=a_me^{-K_mz}+b_me^{K_mz}
\end{align*}
Using eq. 2.21:
\begin{align*}
\Ehat=-\frac{1}{\mu_0\sigma}\frac{d\Bhat_y}{dz}
\end{align*}
See chapter 2: solution for the TM fields.
\begin{align*}
\Ehat_x=-\frac{1}{\mu_0\sigma}\frac{d\Bhat_y}{dz}=\frac{K_m}{\mu_0\sigma_m}\left(a_me^{-K_mz}-b_me^{K_mz}\right)
\end{align*}
From the continuity of the impedance (admittance) follows:

\begin{align*}
\frac{\Bhat_y}{\Ehat_x}=\frac{\mu_0\sigma_m}{K_m g(z)}
\end{align*}
with
\begin{equation}
g(z)=\frac{a_me^{-K_mz}+b_me^{K_mz}}{a_me^{-K_mz}-b_me^{K_mz}}
\label{eq:6-10}
\end{equation}
\subsubsection*{Boundary conditions for TM-fields}
\begin{equation}
K_m g_m^+\rho_m=K_{m+1}g_{m+1}^-\rho_{m+1}
\label{eq:6-bc-1}
\end{equation}
$\Rightarrow$
\begin{equation}
g_m^-=\frac{K_{m+1}g_{m+1}^-\rho_{m+1}+K_m\rho_m\tanh(K_md_m)}{K_m\rho_m+K_{m+1}g_{m+1}^-\rho_{m+1}\tanh(K_md_m)}
\label{eq:6-bc-2}
\end{equation}

From \eqref{eq:6-bc-2} we get $g_1^-$ for the ground surface and from \eqref{eq:6-10} the admittance:

\begin{equation*}
\Bhat_y(0,k,\omega)=\mu_0\sigma_1 C(k,\omega)\Ehat_x(0,k,\omega)
\end{equation*}
with $C=(K_1 g_1^-)^{-1}$

We consider quasi-homogeneous TM-source fields: $K_m^2=i\omega\mu_0\sigma_m$ for all layers. 

We divide \eqref{eq:6-bc-2} by $i\omega\mu_0$:

\begin{align*}
\frac{K_mg_m^+}{i\omega\mu_0\sigma_m}=\frac{g_m^+}{K_m}
\end{align*}
and 
\begin{align*}
\frac{K_{m+1}g_{m+1}^-}{i\omega\mu_0\sigma_{m+1}}=\frac{g_{m+1}^-}{K_{m+1}}
\end{align*}
(see eq. \eqref{eq:6-5} with $\frac{1}{g_m^+}=\frac{K_{m+1}}{K_m}\frac{1}{g_{m+1}^-}$)

or
\begin{align*}
\frac{K_m}{g_m^-}=\frac{K_{m+1}}{g_{m+1}^-}
\end{align*}
according to \eqref{eq:6-5} for TE fields with $g(z)=\frac{1}{G(z)}$

The recursion formula of Wait yields the inverse of $g_1^-$:
\begin{align*}
g_1^-=\frac{1}{G_1^-}
\end{align*}
for $k=0$. And the impedance $z$ of quasi-homogeneous fields:
\begin{align*}
\Ehat_x(0,k,\omega)=z(k,\omega)\Bhat_y(0,k,\omega)
\end{align*}
will be
\begin{align*}
z(k,\omega)=\frac{K_1g_1^-}{\mu_0\sigma_1}=\frac{K_1i\omega}{i\omega\mu_0\sigma_1G_1^-}=i\omega C(k,\omega)
\end{align*}

\subsection{Transfer functions for time-dependent fields}

\begin{itemize}
\item $z(k,\omega)\rightarrow$ Impedance
\item $C(k,\omega)\rightarrow$ Complex penetration depth
\item $Q(k,\omega)\rightarrow$ Potential ratio of internal to external part
\end{itemize}
Transfer functions for TE-fields for $z=0$ and $\sigma=\sigma(z)$:

\begin{align*}
\Ehat_x&=z\Bhat_y && \Ehat_y=-z\Bhat_x\\
\Bhat_z&=C(ik_x\Bhat_x+ik_y\Bhat_y) && I=QE\\
\end{align*}
with $z=i\omega C$ and
\begin{equation}
\begin{split}
Q&=\frac{1-kC}{1+kC}\\
C&=\frac{1}{K_1G_1^-}
\end{split}
\end{equation}
\subsubsection*{Examples}
\begin{enumerate}
\item Homogeneous half space with $\sigma = const \Rightarrow G_1^-=1$, $C(k,\omega)=K^{-1}$ with 
\begin{equation}
K^2=i\omega\mu_0\sigma+k^2
\end{equation}

\item 
\begin{figure}
\begin{center}
\resizebox{0.4\textwidth}{!}
{
\begin{pspicture}(0,-1.358125)(5.4009376,1.358125)
\psline[linewidth=0.04cm](1.3809375,1.1596875)(5.3809376,1.1596875)
\psline[linewidth=0.04cm](1.3809375,-0.4403125)(5.3809376,-0.4403125)
\usefont{T1}{ptm}{m}{n}
\rput(0.41234374,1.1646875){$z=0$}
\usefont{T1}{ptm}{m}{n}
\rput(0.53234375,-0.4353125){$h$}
\usefont{T1}{ptm}{m}{n}
\rput(3.1923437,0.6646875){$\sigma=0$}
\usefont{T1}{ptm}{m}{n}
\rput(3.4823437,-1.1353126){$\sigma=\infty$}
\end{pspicture} 
}
\end{center}
\end{figure}
$K_1=k$, $|K_2|=\infty$, $G_1^-=\frac{1}{\tanh(kh)}$, $C(k)=\frac{1}{k}\tanh(kh)$

For $kp \rightarrow 0$, $kh\rightarrow 0$:

\begin{itemize}
\item \underline{1. Model:}
$C(k,\omega)=C_0(k,\omega)=\frac{p}{1+i}$

..........................????????

\item \underline{2. Model:}
$C(k)\rightarrow C_0=h$
For $kp\rightarrow \infty$, $kh\rightarrow\infty$: $C(k,\omega)=1/k$.


\end{itemize}

\begin{figure}[H]
\begin{center}
\resizebox{0.5\textwidth}{!}
{
\begin{pspicture}(0,-3.2784376)(8.342813,3.2584374)
\psline[linewidth=0.04cm,arrowsize=0.05291667cm 2.0,arrowlength=1.4,arrowinset=0.4]{->}(1.6809375,-0.7615625)(1.6809375,3.2384374)
\psline[linewidth=0.04cm,arrowsize=0.05291667cm 2.0,arrowlength=1.4,arrowinset=0.4]{->}(1.6809375,-0.7615625)(7.6809373,-0.7615625)
\psbezier[linewidth=0.04](1.6809375,2.6384375)(6.8809376,2.6384375)(5.3031597,0.6384375)(7.0809374,-0.3615625)
\psbezier[linewidth=0.04,linestyle=dashed,dash=0.16cm 0.16cm](1.6809375,2.5384376)(6.6809373,2.5384376)(5.3031597,0.5384375)(7.0809374,-0.4615625)
\psline[linewidth=0.04cm](3.5809374,2.5384376)(3.5809374,-0.7615625)
\usefont{T1}{ptm}{m}{n}
\rput(1.0423437,2.9434376){$C(k,\omega)$}
\usefont{T1}{ptm}{m}{n}
\rput(8.032344,-1.0565625){$k$}
\psline[linewidth=0.04cm,arrowsize=0.05291667cm 2.0,arrowlength=1.4,arrowinset=0.4]{->}(2.9809375,-2.8615625)(2.7809374,0.2384375)
\usefont{T1}{ptm}{m}{n}
\rput(4.295469,-3.0565624){C does not depend on k}
\end{pspicture} 
}
\caption{$C$ for Model 1}
\label{fig:c-model1}
\end{center}
\end{figure}

\begin{figure}
\begin{center}
\resizebox{0.5\textwidth}{!}
{
\begin{pspicture}(0,-3.2784376)(8.342813,3.2584374)
\psline[linewidth=0.04cm,arrowsize=0.05291667cm 2.0,arrowlength=1.4,arrowinset=0.4]{->}(1.6809375,-0.7615625)(1.6809375,3.2384374)
\psline[linewidth=0.04cm,arrowsize=0.05291667cm 2.0,arrowlength=1.4,arrowinset=0.4]{->}(1.6809375,-0.7615625)(7.6809373,-0.7615625)
\psbezier[linewidth=0.04](3.5809374,2.5384376)(4.5809374,0.4384375)(5.4809375,0.1384375)(7.2809377,-0.4615625)
\psline[linewidth=0.04cm](3.5809374,2.5384376)(3.5809374,-0.7615625)
\usefont{T1}{ptm}{m}{n}
\rput(1.0423437,2.9434376){$C(k,\omega)$}
\usefont{T1}{ptm}{m}{n}
\rput(8.032344,-1.0565625){$k$}
\psline[linewidth=0.04cm,arrowsize=0.05291667cm 2.0,arrowlength=1.4,arrowinset=0.4]{->}(2.9809375,-2.8615625)(2.7809374,0.2384375)
\usefont{T1}{ptm}{m}{n}
\rput(4.295469,-3.0565624){C does not depend on k}
\psline[linewidth=0.04cm](1.6809375,2.5384376)(3.5809374,2.5384376)
\psline[linewidth=0.04cm,arrowsize=0.05291667cm 2.0,arrowlength=1.4,arrowinset=0.4]{->}(5.9809375,1.4384375)(5.2809377,0.5384375)
\usefont{T1}{ptm}{m}{n}
\rput(6.572344,1.7434375){$1/k$}
\end{pspicture} 
}
\caption{$C$ for Model 2}
\label{fig:c-model2}
\end{center}
\end{figure}
\end{enumerate}
Definition of complex penetration depth for quasi-homogeneous-fields:
\begin{align*}
C_0(\omega)=C(0,\omega),~~~~z_0(\omega)=z(0,\omega)
\end{align*}
with $k|C|^2\ll 1$ or $|C_0|^2\ll \lambda$ for all wave numbers of the source field
\begin{align*}
\Ehat_x=z_0\Bhat_y && \Ehat_y=...\\
\Bhat_z=C_0(...) &&with~~ |\Bhat_z|\ll|\Bhat_{xy}|
\end{align*}
This is also valid for TM-fields with admittance $A(k,\omega)$ as transfer function (from eq. 6.13)
\begin{equation}
\begin{split}
\Bhat_x=A(k,\omega)\Ehat_y\\
\Bhat_y=A(k,\omega)\Ehat_x\\
\Ehat_z=C(k,\omega)(ik_x\Ehat_x+ik_y\Ehat_y)
\end{split}
\end{equation}
with $A=\mu_0\sigma C$ and $C=\frac{1}{K_1g_1^-}$ for layered ground.

At all depth ranges $\omega\mu_0\sigma\gg k^2$ and therefore $K_m=\sqrt{i\omega\mu_0\sigma_m}$ and $C=(K_1G_1^-)^{-1}$

\begin{align*}
C_0^{(TE)}=\frac{1}{K_1G_1^-}&& C_0^{(TM)}=\frac{G_1^-}{K_1}
\end{align*}

\begin{equation}
i\omega\mu_0\sigma C_0^{(TE)}C_0^{(TM)}=1
\end{equation}

\subsection{Complex penetration depth $C$ for simple conductivities}

\subsubsection*{Model 1, 2}
!!!!!!!!!!!!!!!!!!!!model 1, mode2 bilder

$C(k,\omega)=K^{-1}\rightarrow\frac{p}{1+i}$

$K_2=\infty$ and 

\begin{equation}
G_1^-=\frac{1}{\tanh(K_1h}
\end{equation}

$C(k,\omega)=\frac{\tanh(K_1h}{K_1}\stackrel{kp\rightarrow 0}\rightarrow\frac{p_1}{1+i}\tanh\left(\frac{(1+i)h}{p_1}\right)$

for $h\ll p_1 \Rightarrow C_0=h \Rightarrow$ compare with eq. 6.18

\subsubsection*{Model 3}

$K_2=k$
For $|K_1|\gg k (\rho k\ll 1) \Rightarrow G_1^-=\tanh(K_1d)$

$C(k,\omega)=\frac{\coth(K_1d}{K_1}$ !!!!!!!!!!!!!!!????????

....
$C_0=\frac{1}{3}d+\frac{p_1^2}{2id}$

\subsubsection*{Representation of $C(\omega)$ with $T=\frac{2\pi}{\omega}$ as a parameter for models 1,2,3}

\begin{figure}[H]
\begin{center}
\resizebox{0.6\textwidth}{!}
{
\begin{pspicture}(0,-3.308125)(11.522813,3.308125)
\psline[linewidth=0.04cm,arrowsize=0.05291667cm 2.0,arrowlength=1.4,arrowinset=0.4]{->}(1.7809376,2.8096876)(1.7809376,-3.1903124)
\psline[linewidth=0.04cm,arrowsize=0.05291667cm 2.0,arrowlength=1.4,arrowinset=0.4]{->}(1.7809376,2.8096876)(9.780937,2.8096876)
\usefont{T1}{ptm}{m}{n}
\rput(10.622344,3.1146874){$-Im(C_0)$}
\usefont{T1}{ptm}{m}{n}
\rput(0.76234376,-3.0853126){$Re(C_0)$}
\psline[linewidth=0.04cm](1.7809376,2.8096876)(9.580937,-1.7903125)
\usefont{T1}{ptm}{m}{n}
\rput(9.855938,-0.9853125){Model 1}
\psline[linewidth=0.04cm,arrowsize=0.05291667cm 2.0,arrowlength=1.4,arrowinset=0.4]{->}(5.1809373,0.6096875)(5.9809375,0.1096875)
\usefont{T1}{ptm}{m}{n}
\rput(6.162344,-0.0853125){$T$}
\psbezier[linewidth=0.04,linestyle=dashed,dash=0.16cm 0.16cm](1.7809376,2.8096876)(4.9809375,1.0096875)(5.5809374,-0.6903125)(1.8809375,-2.1903124)
\usefont{T1}{ptm}{m}{n}
\rput(3.3704689,-2.1853125){Model 2}
\psbezier[linewidth=0.04,linestyle=dashed,dash=0.16cm 0.16cm](1.8809375,2.7096875)(3.6809375,1.6096874)(5.0809374,1.5763541)(8.680938,1.3874652)
\usefont{T1}{ptm}{m}{n}
\rput(8.862968,1.0146875){Model 3}
\end{pspicture} 
}
\caption{Real and Imaginary part of $C$ in dependency of $T$}
\label{fig:crealimag}
\end{center}
\end{figure}

\subsubsection*{Model 4}
\begin{figure}[H]
\begin{center}
\resizebox{0.4\textwidth}{!}
{
\begin{pspicture}(0,-1.008125)(5.0009375,1.008125)
\psline[linewidth=0.04cm](0.9809375,0.8096875)(4.9809375,0.8096875)
\psline[linewidth=0.04cm](0.9809375,-0.1903125)(4.9809375,-0.1903125)
\usefont{T1}{ptm}{m}{n}
\rput(0.41234374,0.8146875){$z=0$}
\usefont{T1}{ptm}{m}{n}
\rput(0.23234375,-0.1853125){$h$}
\usefont{T1}{ptm}{m}{n}
\rput(3.6723437,0.2146875){$\sigma_1=0$}
\usefont{T1}{ptm}{m}{n}
\rput(3.4823437,-0.7853125){$\sigma_2$}
\end{pspicture} 
}
\caption{Model 4}
\label{fig:6-model4}
\end{center}
\end{figure}

$K_1=k$ for $|K_2|\gg k \Rightarrow G_1^-=\frac{K_2}{k+K_2\tanh(kh)}$ and
\begin{align*}
C(k,\omega)=\frac{k+K_2\tanh(kh)}{K_2k}\stackrel{kh\rightarrow 0}\rightarrow=h+\frac{p_2}{1+i}=h+\frac{p_2}{2}-\frac{ip_2}{2}
\end{align*}

\subsubsection*{Model 5}
\begin{figure}[H]
\begin{center}
\resizebox{0.4\textwidth}{!}
{
\begin{pspicture}(0,-1.008125)(5.9828124,1.008125)
\psline[linewidth=0.04cm](0.9809375,0.8096875)(4.9809375,0.8096875)
\psline[linewidth=0.04cm](0.9809375,-0.1903125)(4.9809375,-0.1903125)
\usefont{T1}{ptm}{m}{n}
\rput(0.41234374,0.8146875){$z=0$}
\usefont{T1}{ptm}{m}{n}
\rput(0.23234375,-0.1853125){$d$}
\usefont{T1}{ptm}{m}{n}
\rput(3.9423437,0.2146875){$\sigma_1d=\tau$}
\usefont{T1}{ptm}{m}{n}
\rput(4.302344,-0.7853125){$\sigma_2\ll \sigma_1$}
\end{pspicture} 
}
\caption{Model 5}
\label{fig:6-model5}
\end{center}
\end{figure}
$d\ll p_1$ for $|K_1|\gg |K_2| \Rightarrow$
\begin{align*}
G_1^-=\frac{K_2+K_1\tanh(K_1d)}{K_1}
\end{align*}
\begin{align*}
C(k,\omega)=\frac{1}{K_2+K_1\tanh(K_1d)}\stackrel{kp\rightarrow 0}\rightarrow \frac{1}{K_2+K_1^2d}=\frac{C_2}{1+i\omega\mu_0\tau C_2}
\end{align*}

with $C_2=K_2^{-1}=\frac{p_2}{1+i}$. 

Phase of $C$: $[-45^\circ,-90^\circ$]

Phase of $z$: $[45^\circ,0^\circ$]

\begin{figure}[H]
\begin{center}
\resizebox{0.6\textwidth}{!}
{
\begin{pspicture}(0,-3.3573437)(11.522813,3.3573437)
\psline[linewidth=0.04cm,arrowsize=0.05291667cm 2.0,arrowlength=1.4,arrowinset=0.4]{->}(1.7995313,2.760469)(1.7995313,-3.239531)
\psline[linewidth=0.04cm,arrowsize=0.05291667cm 2.0,arrowlength=1.4,arrowinset=0.4]{->}(1.7995313,2.760469)(9.799531,2.760469)
\usefont{T1}{ptm}{m}{n}
\rput(10.622344,3.0654688){$-Im(C_0)$}
\usefont{T1}{ptm}{m}{n}
\rput(0.76234376,-3.1345313){$Re(C_0)$}
\psline[linewidth=0.04cm](1.7995313,2.760469)(9.599531,-1.8395312)
\usefont{T1}{ptm}{m}{n}
\rput(9.849532,-1.0345311){Model 1}
\psline[linewidth=0.04cm,arrowsize=0.05291667cm 2.0,arrowlength=1.4,arrowinset=0.4]{->}(5.199531,0.56046885)(5.9995313,0.06046885)
\usefont{T1}{ptm}{m}{n}
\rput(6.162344,-0.13453116){$T$}
\psbezier[linewidth=0.04,linestyle=dashed,dash=0.16cm 0.16cm](1.76,0.6589062)(3.56,0.6589062)(6.56,0.55890626)(9.16,-2.5410938)
\usefont{T1}{ptm}{m}{n}
\rput(3.378594,-2.2345312){Model 2}
\psbezier[linewidth=0.04,linestyle=dashed,dash=0.16cm 0.16cm](6.56,2.7589064)(5.76,1.1589062)(6.36,0.25890625)(9.86,-1.3410938)
\usefont{T1}{ptm}{m}{n}
\rput(8.863593,0.9654688){Model 3}
\usefont{T1}{ptm}{m}{n}
\rput(6.711406,3.1639063){$1/\omega\mu_0\tau$}
\usefont{T1}{ptm}{m}{n}
\rput(1.4114063,0.6639063){$h$}
\end{pspicture} 
}
\caption{Real and Imaginary part of $C$ in dependency of $T$ for Model 4 and 5}
\label{fig:crealimag2}
\end{center}
\end{figure}


\setcounter{section}{5}\setcounter{subsection}{1}\addtocounter{subsection}{-1}\subsection{Field station }

\begin{itemize}
\tightlist
\item
  3 component magnetometer
\item
  For long periods \(10^{-6}-1 \physu{Hz}\) use of flux gate
  magnetometers
\item
  For short periods \(10^{-3}-10^3 \physu{Hz}\) use induction coil
  magnetometers
\item
  Non-polarizable electrodes
\item
  e.g.~Ag-AgCl have nearly no disturbing potential
\end{itemize}

\setcounter{section}{5}\setcounter{subsection}{2}\addtocounter{subsection}{-1}\subsection{Analysis of MT time series }

Overview of different steps:

Choice of sensors \(\to\) Observation \(\to\) Pretreatment of time
series in time domain \(\to\) Transform the data to frequency domain
\(\to\) Response curve \(\to\) Power spectra, stacking \(\to\)
Calculation of the transfer functions
(\(Z_{xx}, Z_{xy}, Z_{yx}, Z_{yy}, T_x, T_y\))

\begin{enumerate}
\def\labelenumi{\arabic{enumi}.}
\item
  Choice of the frequency range and magnetometer: indution coil or flux
  gate? Determination of the sampling interval \(\Delta t\)
\item
  Observation. Measure \(E_x(t)\), \(E_y(t)\), \(H_x(t)\), \(H_y(t)\),
  \(H_z(t)\)

  \begin{itemize}
  \tightlist
  \item
    Period: 1 d \ldots{} 1 yr
  \item
    Digitalisation \(H(t_j) \to x_j\), \(E(t_j) \to z_j\)
  \end{itemize}
\item
  Choice of analysing intervals:

  \begin{itemize}
  \tightlist
  \item
    Intervals with strong magnetic activity but without visible noise
  \item
    Choose time intervals suitable for FFT
  \end{itemize}
\item
  Data processing in time domain

  \begin{itemize}
  \tightlist
  \item
    Determine the trend of each chosen time interval and substract it
  \item
    Numerical filter

    \begin{itemize}
    \tightlist
    \item
      Aim: distinguish or emphasize suitable frequency bands
    \item
      Preparation for the frequency analysis
      \[ y_n = \sum_{n'=-N}^{N} w_{n'}x_{n-n'} \] high pass, low pass
      filters. Consider low pass filter, e.g.
    \item
      rectangle/sinc LP filter \[
         \tilde{w}(\omega) = \begin{cases}
                1   & |\omega|<\omega_0
           \\ (1/2  & |\omega|=\omega_0 )
            \\  0   & \text{otherwise}.
            \end{cases}
       \] This has difficulties in practice due to strong oscillations
      in the vicinity of cutoff frequency.
    \item
      Trapezoidal LP filter \[
         \tilde{w}(\omega) = \begin{cases}
                1   & 0<|\omega|<\omega_1
              \frac{\omega_2-\omega}{\omega_2-\omega_1}
           \\       & \omega_1 < |\omega| < \omega_2
            \\  0   & \text{otherwise}.
            \end{cases}
       \] The following quantities are introduced as filter parameters:
      \[
         \omega_0 = \tfrac12(\omega_1 + \omega_2)
       \] \[
         \Omega = \tfrac12(\omega_2 - \omega_1).
       \] The \emph{steepness} of the filter is defined by \[
         S = \frac{\omega_0}{\Omega}.
       \] The filter function \(w(t)\) can be obtained by applying the
      FT on \(\tilde{w}(\omega)\) \[
         \omega(t) = \tfrac1\pi \int\limits_0^{\infty}\ttd{\omega} \tilde{w}\cos(\omega t)
                   = \frac{\omega_0}{\phi}\frac{\sin(\omega_0 t)}{\omega_0t}
                       \cdot\frac{\sin(\Omega t)}{\Omega t}.
       \] The filter should have finite length, so define
      \(\tilde{w}(\omega)\): \[
         \tilde{w}_m(\omega) = \tilde{w}_0 + 2\sum_{n=1}^K w_n\cos(2\pi nm/N)
       \] with \[
         \tilde{w}_0 = \frac{\omega_0}{\omega\rmsc{ny}}
             \left(1 + 2\frac{\sin\alpha_n}{\alpha_n}\cdot\frac{\sin\beta_n}{\beta_n}\right)
       \]
    \item
      Trapezoidal HP filter \[
         \tilde{w}\rmsc{H} = 1 - \tilde{w}\rmsc{L}.
       \] Instead of high pass, a polynomial equation can be applied.
      (???) \[
         y_n' = \frac{d}{T}(t - t_0 - \tfrac{\tau}{2})
       \] with \(d = y(t_N) - y(t_0)\). The analysis can be realized in
      small and overlapping frequency ranges. Example of typical filter
      parameters
    \item
      Magnetic storm, \(T = 8\physu{d}\). \(\Delta t\sim 1 h\). Low pass
      \(0.75\physu{cpd}\), polynomial high pass.
    \item
      Day variation, \(T = 1\physu{d}\). \(\Delta t\sim 1 h\). Low pass
      \(4\physu{cpd}\), polynomial high pass.
    \item
      Variations, \(T = 6\physu{h}\). \(\Delta t\sim 1 \physu{min}\).
      Low pass \(2\physu{cpd}\), polynomial high pass.
    \item
      Variations, \(T = 6\physu{h}\). \(\Delta t\sim 1 \physu{min}\).
      Low pass \(6\physu{cpd}\), high pass \(0.5\physu{cpd}\).
    \item
      Variations, \(T = 6\physu{h}\). \(\Delta t\sim 1 \physu{min}\).
      Low pass \(15\physu{cpd}\), high pass \(1.5\physu{cpd}\).
    \end{itemize}
  \end{itemize}
\item
  Data processing in frequency domain

  \begin{itemize}
  \tightlist
  \item
    Harmonic analysis of the filtered time series \[
       x_j , j\in\{0\ldots N\}
        \Rightarrow \tilde{X}_m, m\in\{0\ldots M/2\}
     \] Outcome: raw spectra. \[
       x_j = \sum_{m=0}^{M} (a_m\cos(m\phi_j) + b_m\sim(m\phi_j) + S x_j
     \] with \(\phi_j = \frac{2\pi j}{N}\). \[
       a_m = \frac{2}N \sum_{j=0}^{N-1} x_j\cos(m\phi_j)
     \] \[
       b_m = \frac{2}N \sum_{j=0}^{N-1} x_j\sin(m\phi_j)
     \] \[
       a_0 = \tfrac1N \sum_{j=0}^{N-1} x_j
     \] \[
       a_M = \tfrac1N \sum_{j=0}^{N-1} x_j(-1)^j
     \] The corresponding frequency \[
       \omega_m = \frac{2\pi}{T} = \frac{\pi}\Delta
     \] is called the Nyqvist frequency. The analysis for
    \(\omega > \omega_M\) is not unique, we get \emph{aliasing}. In the
    harmonic analysis only those frequencies are considered which are
    less than the Nyqvist frequency. It is assumed that the time series
    vanishes outside the transformation interval, but this must be
    enforced by multiplication with a \emph{window function}. This is
    equivalent to a convolution in frequency domain. Example of a window
    functions:

    \begin{itemize}
    \tightlist
    \item
      Hann, \(w(t) = \tfrac12 (1 + \cos\tfrac{2\pi}{T}(t - t_0 - T/2))\)
      Different possibilities to calculate harmonic coefficients.
    \end{itemize}
  \item
    Calibration: at this stage the response curve of the device is
    considered. For example, with
  \item
    \(g(t)\) observed time series
  \item
    \(f(t)\) true time series
  \item
    \(r(t)\) response function of the device \[
    g(t) = r(t) \ast f(t)
       \] Example: measurement of the earth magnetic field pulsation
    using an induction coil magnetometer.
  \item
    Time dom. \(g(t) = r(t)\ast \partial_t f(t)\)
  \item
    Freq dom. \(G(\omega) = i\omega F(\omega)\)
  \end{itemize}
\end{enumerate}


