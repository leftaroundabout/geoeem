\section{Basic equations for EM
methods}\label{basic-equations-for-em-methods}

Decay time of free space charges follows an exponential law
$\eta(t) = \eta_0 e^{-t/\tau}$, where \setcounter{equation}{2}\[
  \tau = \frac{\epsilon \epsilon_0}{\sigma}
\] - For air: $\sigma \approx 10{-14}\physu{S/m}$, $\epsilon=1$ thus
$\tau \approx 15\physu{min}$. - For subsurface:
$\sigma \approx 10{-5}\physu{S/m}$, $\epsilon=1..80$ thus
$\tau \approx 10^{-4}\physu{s}$.

Equtions for - Static fields -- no induction - in time scale of $\tau$
slowly oscillating fields -- quasi-stationary fields - Induction by
induced currents much larger than displacement currents - in time scale
of $\tau$ quickly oscillating fields -- displacement currents

\begin{verbatim}
                  | Static          | Slow           | Quick
\end{verbatim}

$\nabla\times\Magn$ \textbar{} $\mu_0\jCurr$ \textbar{} $\mu_0\jCurr$
\textbar{} $\mu_0\epsilon\epsilon_0\dot E$ $\nabla\times\Elct$
\textbar{} $0$ \textbar{} $-\dot\Magn$ \textbar{} $-\dot\Magn$

\subsection{General Maxwell equations for electric and magnetic field
quantities}\label{general-maxwell-equations-for-electric-and-magnetic-field-quantities}

\setcounter{equation}{3}\[\tag{3.a}
  \nabla\times\Elct  = -\dot{\Magn}
\] Ampere's law states that every closed loop of current will have an
associated magnetic field of magnitude proportional to the total current
flow: \setcounter{equation}{3}\[\tag{3.b}
  \nabla\times\Magn  = \mu_0(\sigma\Elct + \partial_t\Elct)
\] \setcounter{equation}{3}\[\tag{3.c}
  \nabla\cdot\Magn   = 0
\] \setcounter{equation}{3}\[\tag{3.d}
  \nabla\cdot\Elct   = \eta/\epsilon_0
\]

Rotation of (2.3a) gives \[
  \nabla\times\nabla\times\Elct = -\nabla\times\Magn
\] Time-deriv of (2.3b) gives \[
  \nabla\times\partial_t\Magn = \mu_0(\sigma\dot\Elct + \epsilon_0\ddot\Elct).
\] Without sources in the conductive subsurface
$\nabla\cdot\jCurr = \nabla\cdot(\sigma\Elct) = 0$. Using the double
rotation identity, it follows for $\sigma\gg\omega\epsilon$: \setcounter{equation}{6}\[
  \nabla^2\Field = \mu_0 \sigma \dot\Field
\] where $\Field$ can be eithe electric or magnetic field. This is a
diffusion equation for the fields coming from the external sources and
diffusing through the earth. Thus EM fields propagate diffusively; our
measurements yield \emph{volume} soundings (response functions are
volumetric averages of the sample medium).

OTOH, for quickly oscillating fields: \setcounter{equation}{7}\[
  \nabla^2\Field = \mu_0\epsilon\epsilon_0\ddot\Field.
\]

For the quasi-static fields in non-conductors: \[
  \nabla\times\Magn = 0
\] $\Magn$ can be represented as gradient of a scalar potential: \[
  \Magn = -\nabla U
\] with $\nabla^2 U = 0$.

Application areas:

\begin{itemize}
\itemsep1pt\parskip0pt\parsep0pt
\item
  Atmosphere:

  \begin{itemize}
  \itemsep1pt\parskip0pt\parsep0pt
  \item
    For $T\gg15\physu{min}$, field variations diffuse through the
    atmosphere.
  \item
    In smaller time scales, they travel as waves.
  \end{itemize}
\end{itemize}

\subsection{Fourier transform of the field
quantities}\label{fourier-transform-of-the-field-quantities}

Switch to (position-space) frequency domain: \[
  \Magn(\posr,t) \to \tilde\Magn(\posr, \omega)
\] where $\omega=\frac{2\pi}T$ is the angular frequency, $\tilde\Magn$ a
complex Fourier amplitude. \[
  \tilde f(\omega) = \IRint\ttd{t}\: f(t)e^{-i\omega t}
\] \[
  f(t) = \frac1{2\pi}\IRint\ttd{t}\: \tilde f(\omega)e^{i\omega t}
\]

Induction equations in $(\posr,\omega)$ domain with $n=1$: \setcounter{equation}{10}\[
  \nabla^2\tilde\Field = i\omega\mu_0\sigma\tilde\Field
\] Wave equations: \[
  \nabla^2\Field = -\mu_0\epsilon\epsilon_0\omega^2\tilde\Field.
\] We scale (2.10) with the skin-depth formula: \[
  p = \sqrt{\frac{2}{\omega\mu_0\sigma}}
\] which increases with $\sqrt{\rho T}$ \[
  \nabla^2\Field = (\tfrac{1+i}{p})^2\tilde\Field
\] The wave equation can be expressed in terms of the speed of light
$c = (\mu_0\epsilon\epsilon_0)^{-1/2}$ \[
  \nabla^2\tilde{Field}
\]

Estimation of $p$ in km: - $T$ in s (Pulsations):
$p = \tfrac12\sqrt{\rho/\Ohmm \cdot T/\physu{s}}$

Now switch also to wavenumber domain (in cartesian coordinates): \[
  \posr = x \unitV{x} + y\unitV{y} + z\unitV{z}
\] Assume in the following \[
  \wavnum = k\rmsc{x}\unitV{x} + k\rmsc{y}\unitV{y}
\] \[
  k = \sqrt{k\rmsc{x}^2 + k\rmsc{y}^2}
\] 2D Fourier transform of the induction eqn: \setcounter{equation}{13}\[
  \nabla^2\hat\Field = -k^2 \hat\Field + \frac{\ttd^2\Field}{\ttd z^2}
\] \[
  \frac{\ttd^2\hat{\Field_x}}{\ttd z^2} = (i\omega\mu_0\sigma + k^2)\hat\Field_x
\] We introduce the \emph{vertical wave number} \[(1+14)
  K = \sqrt{i\omega\mu_0\sigma + k^2} = \sqrt{(\tfrac{1+i}{p})^2 + k^2}
\] in a layer of conductivity $\sigma$. $C = K^{-1}$ is called
\emph{complex penetration depth}.

Make further approximations: for $(p\cdot k)^2 \gg 1$, \[
  K \approx k(1+\tfrac{i}{(pk)^2}
\] which means the field diffuses through the layer like it would
through a non-conductor. For $(p\cdot k)^2 \ll 1$, \[
  K \approx \frac{1+i}{p}(1 - \tfrac{i}4(pk)^2)
\] which means the field diffuses like a quasi-homogeneous field
($k=0$).

\subsection{Example: oscillating current in height
$H=300\physu{km}$}\label{example-oscillating-current-in-height-h300physukm}

\begin{itemize}
\itemsep1pt\parskip0pt\parsep0pt
\item
  For $T = 36\physu{s}$, $p = 30\physu{km}$ so $p\cdot k = 0.15$,
  i.e.~diffusion is dictated by $p$
\item
  For $T = 1\physu{h}$, $p = 300\physu{km}$ so $p\cdot k = 1.5$, which
  permits neither approximation
\item
  For $T = 24\physu{h}$, $p = 1500\physu{km}$, diffusion is dictated by
  $k$.
\end{itemize}

\section{Induction in homogeneous half
space}\label{induction-in-homogeneous-half-space}

Solutions in $z>0$ and $z<0$ can be combined by continuity equations.

\begin{enumerate}
\def\labelenumi{\arabic{enumi}.}
\itemsep1pt\parskip0pt\parsep0pt
\item
  Solution for tangential-electric source field (e.g.~plane layer
  currents in the ionosphere) $E\rmsc{z}=0$

  \begin{itemize}
  \itemsep1pt\parskip0pt\parsep0pt
  \item
    Induced currents are also plane layer currents
  \item
    $E\rmsc{z} = 0$ in the whole space (\emph{tangential electric}
    fields).
  \end{itemize}
\end{enumerate}

For $z>0$ solution of eq. 2.13
$\frac{\ttd^2\hat{E}\rmsc{x}}{\ttd z^2} = (i\omega\mu_0\sigma + k^2)\hat{E}\rmsc{x}$
\[
  \hat{E}\rmsc{x}(z,k,w) = A\rmsc{x} e^{-Kz} + B\rmsc{x}e^{Kz}
\] \[
  \hat{E}\rmsc{y}(z,k,w) = A\rmsc{y} e^{-Kz} + B\rmsc{y}e^{Kz}
\] Constraint: $E\rmsc{x}\to 0$ as $z \to\infty$

Derivation of the ma/detic field from $\Elct$ using \[
  \nabla\times\tilde\Elct = -i\omega\tilde\Magn
\] \[
  \nabla\times\tilde\Elct
    = \left( \partial_y\tilde E\rmsc{z} - \partial_z \tilde E\rmsc{y} \right) \unitV{x}
    + \left( \partial_z\tilde E\rmsc{x} - \partial_x \tilde E\rmsc{z} \right) \unitV{y}
    + \left( \partial_x\tilde E\rmsc{y} - \partial_y \tilde E\rmsc{x} \right) \unitV{z}
    = \left( - \partial_z \tilde E\rmsc{y} \right) \unitV{x}
    + \left( \partial_z\tilde E\rmsc{x}  \right) \unitV{y}
    + \left( \partial_x\tilde E\rmsc{y} - \partial_y \tilde E\rmsc{x} \right) \unitV{z}
\] \[
  \nabla\times\hat\Elct
     = -\frac{\ttd\hat E\rmsc{y}}{\ttd z}\unitV{x}
       + \frac{\ttd\hat E\rmsc{x}}{\ttd z}\unitV{y}
       + (ik\rmsc{x} \hat E\rmsc{y} - ik\rmsc{y}\hat E\rmsc{x})
\] \setcounter{equation}{15}\[
  \hat B\rmsc{x} = \frac{1}{i\omega}\frac{\ttd\hat E\rmsc{y}}{\ttd z}
\] \[
  \hat B\rmsc{y} = -\frac{1}{i\omega}\frac{\ttd\hat E\rmsc{x}}{\ttd z}
\] \[
  \hat B\rmsc{z} = \tfrac1\omega (k\rmsc{y}\hat E\rmsc{x} - k\rmsc{x}\hat E\rmsc{y})
\]

For $-H < z < 0$: solution of (2.5) (Helmholtz) \[
  \hat E\rmsc{x} = a\rmsc{x} e^{-K_0 z} + b\rmsc{x} e^{K_0 z}
\] \[
  \hat E\rmsc{y} = a\rmsc{y} e^{-K_0 z} + b\rmsc{y} e^{K_0 z}
\] with \[
  K_0 = \sqrt{i\omega\mu_0\sigma + k^2 - k\rmsc{EM}}
\] and \setcounter{equation}{16}\[
  k\rmsc{EM} = \omega\sqrt{\mu_0\epsilon\epsilon_0} = \frac{\omega}{c}
\] $K_0$ is the wavenumber of an EM wave propagating with the velocity
of light $c$ \[
  c = \frac{\lambda\rmsc{EM}}{T} = \frac{\omega}{k\rmsc{EM}}.
\]

Example contributions to $K_0$: \[
  \sigma_0 = 10^{-14}\physu{S/m} \thus p_0 \geq \frac12 10^2\physu{km}
\] for $T>1\physu{s}$.

\[
  \lambda = 2\pi R\rmsc{E} \thus k\geq R_E^{-1} \approx \frac1{6000}\physu{km^{-1}}
\] $p_0k \geq \frac{10^7}{12000}$, thus diffusion through the atmosphere
is like through a conductor with $\sigma = 0$.
